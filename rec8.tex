\documentclass[10pt, handout, xcolor=table]{beamer}

\usepackage[utf8]{inputenc}
\usepackage{amsmath}
\usepackage{amsfonts}
\usepackage{amssymb}
\usepackage{mathtools}
\newcommand*\themecol{\usebeamercolor[fg]{structure}}

\setbeamertemplate{navigation symbols}{}
 \setbeamertemplate{footline}[frame number]

\newcommand{\overbar}[1]{\mkern 1.5mu\overline{\mkern-1.5mu#1\mkern-1.5mu}\mkern 1.5mu}

\usepackage{tikz}
\usetikzlibrary{shapes.geometric, arrows}
\tikzstyle{prob} = [rectangle, minimum width=3cm, text width = 4.5cm, minimum height=1cm, text centered, draw=black, fill= blue!20]
\tikzstyle{stat} = [rectangle, minimum width=3cm,  text width = 4.5cm, minimum height=1cm, text centered, draw=black, fill= red!20]
\tikzstyle{arrow} = [thick,->,>=stealth]

\DeclarePairedDelimiter\abs{\lvert}{\rvert}%
\DeclarePairedDelimiter\norm{\lVert}{\rVert}
\DeclarePairedDelimiter\ceil{\lceil}{\rceil}
\DeclarePairedDelimiter\floor{\lfloor}{\rfloor}



\setlength{\parindent}{0pt}
\setlength{\parskip}{6pt}


\title{STAT 111\\
{\small Recitation 8}}

\author{Mo Huang}
\institute{Email: mohuang@wharton.upenn.edu \\
\vspace{0.25cm}
Office Hours: Wednesdays 3:00 - 4:00 pm, JMHH F96\\
\vspace{0.25cm}
Slides (adapted from Gemma Moran): \url{github.com/mohuangx/STAT111-Fall2018} }


\date{November 2, 2018}


\begin{document}

\begin{frame}
\titlepage
\end{frame}

\begin{frame}{Introduction to Hypothesis Testing}

\begin{itemize}\itemsep3ex
\item {\themecol Statistics} is {\em inductive} (or ``bottom-up logic''). We start with finite observations and then attempt to make objective statements about the world or \textbf{test hypotheses}. 
\item For example: Is this coin fair? Is this medicine more effective?
\item \textbf{Null hypothesis $(H_0)$:} A general statement indicating no significant difference or phenomenon.
\begin{itemize}
\item For example: This coin is fair. This medicine is equally as effective.
\end{itemize}
\item \textbf{Alternative hypothesis $(H_1)$:} The hypothesis used that is contrary to the null hypothesis.
\begin{itemize}
\item For example. This coin is unfair. This medicine is more effective.
\end{itemize}
\item \textbf{Basic premise of hypothesis testing:} Observe a random sample from a population. If the sample is consistent with $H_0$, do not reject $H_0$. If the sample is inconsistent or unlikely under $H_0$, then reject $H_0$ in favor of $H_1$.
\end{itemize}

\end{frame}

\begin{frame}{Approach to hypothesis testing}

\begin{enumerate}\itemsep4ex
\item Define the null hypothesis $H_0$ and the alternative hypothesis $H_1$ in terms of parameters.
\item Choose a type I error $\alpha$.
\item Determine the test statistic.
\item Calculate the critical region/p-value.
\item Reject or not.
\end{enumerate}

\end{frame}

\begin{frame}{1. Null and alternative hypotheses}

\begin{itemize}\itemsep3ex
\item Null hypotheses are most often single values. In the fair coin example, $H_0: \theta = 0.5$.
\item Alternative hypotheses are typically ranges and can be {\themecol one-sided} or {\themecol two-sided}.
\item For example:
\begin{itemize}
\item One-sided: The coin is more likely to favor heads $(H_1: \theta > 0.5)$.
\item Two-sided: The coin is unfair $(H_1: \theta \neq 0.5)$.
\end{itemize}
\item Whether the alternative is one-sided or two-sided affects the probability calculation.
\end{itemize}

\end{frame}

\begin{frame}{2. Choose a type I error $\alpha$}

\begin{itemize}\itemsep3ex
\item The type I error $\alpha$ is the probability of rejecting the null hypothesis when the null hypothesis is true. Can think of this as a mistaken rejection.
\item The experimenter can set $\alpha$ depending on how willing they are to mistakenly reject $H_0$. Most common values are $0.01$ and $0.05$.
\item For experiments where we want to be really sure we are not making a mistake, we would want $\alpha$ to be small.
\item If $\alpha = 0.05$, this means that in roughly 5 out of every 100 repetitions of the experiment, we will mistakenly reject $H_0$.
\end{itemize}

\end{frame}

\begin{frame}{3. Determining the test statistic}

\begin{itemize}\itemsep4ex
\item The test statistic is a numerical function of the data we use to reject or accept the null hypothesis.
\item Most often, it is the estimate of the parameter.
\item For example, the test statistic for the fair coin flip example ($H_0: \theta = 0.5$) is the proportion of heads.
\item We can also use the number of heads, which would involve a different probability calculation.
\end{itemize}

\end{frame}

\begin{frame}{4-5. Determining significance and rejection}

\begin{itemize}
\item Two equivalent approaches:
\begin{enumerate}
\item Determine critical region/point at level $\alpha$ and reject $H_0$ if test statistic falls inside critical region.
\item Calculate p-value of the test statistic and reject $H_0$ if less than $\alpha$.
\end{enumerate}
\item Approach 1
\begin{itemize}
\item The \textbf{critical region} is calculated such that the probability the test statistic falls in the critical region is $\alpha$ if $H_0$ is true.
\item One-sided: $P(X \geq A) = \alpha$ or $P(X \leq A) = \alpha$.
\item[] Two-sided: $P(X \leq A) = P(X \geq B) = \alpha/2$.
\item Reject $H_0$ if test statistic falls inside critical region.
\end{itemize}
\item Approach 2
\begin{itemize}
\item The \textbf{p-value} is the probability under $H_0$ of obtaining a result of more or equal extremeness than the observed test statistic.
\item One-sided: p-value = $P(X \geq x)$ or $P(X \leq x)$.
\item[] Two-sided: p-value = $2P(X \geq x)$ or $2P(X \leq x)$ depending on if $x$ is greater than or less than the mean.
\item Reject $H_0$ if p-value is less than $\alpha$.
\end{itemize}
\end{itemize}

\end{frame}

\begin{frame}{Example}
\begin{itemize}
\setlength{\itemsep}{5pt}
\item I want to test the hypothesis a coin is unbiased. I observed 1,070 heads out of 2,000 tosses. Let $\theta$ be the probability of tossing a head. Using approach I, calculate the critical point(s) and decide whether to reject $H_0$ at $\alpha = 0.05$.
\footnotesize
\item<2->[Ans:] \color{red} $H_0: \theta = 0.5$ vs. $H_1: \theta \neq 0.5$. Two-sided test.
\item<2->[] \color{red} Test-statistic: $X$ is the number of heads tossed. $x = 1070$.
\item<2->[] \color{red} Under $H_0$, $\mu = n\theta = 2000(0.5) = 1000$ and $\sigma^2 = n\theta(1-\theta) = 500$.
\item<2->[] \color{red} We want to find critical points $A$ and $B$ at $\alpha = 0.05$:
\begin{align*}
P(X \leq A) = 0.025 &\text{ and } P(X \geq B) = 0.025\\
P\left(\frac{X - 1000}{\sqrt{500}} \leq \frac{A - 1000}{\sqrt{500}}\right) =0.025 &\text{ and } P\left(\frac{X - 1000}{\sqrt{500}} \geq \frac{B - 1000}{\sqrt{500}} \right)=0.025 \\
P\bigg(Z \leq \frac{A - 1000}{\sqrt{500}}\bigg) = 0.025  &\text{ and } P\bigg(Z\geq \frac{B - 1000}{\sqrt{500}}\bigg) = 0.025\\
\end{align*}
\vspace*{-7ex}
\item<2->[] \color{red} Looking at the z-chart for $z$ where the probabilities are 0.025 and 0.975, we see that
\[
\frac{A-1000}{\sqrt{500}} = -1.96 \text{ and } \frac{B - 1000}{\sqrt{500}} = 1.96
\]
\item<2->[] \color{red} Solving for $A$ and $B$, we get $A = 956$ and $B = 1,044$. Since 1,070 is in the critical region, we reject $H_0$.
\end{itemize}
\end{frame}

\begin{frame}{Example}
\begin{itemize}
\setlength{\itemsep}{5pt}
\item I want to test the hypothesis a coin is unbiased. I observed 1,070 heads out of 2,000 tosses. Let $\theta$ be the probability of tossing a head. Calculate the p-value and decide whether to reject $H_0$ at $\alpha = 0.05$.
\item<2->[Ans:] \color{red} $H_0: \theta = 0.5$ vs. $H_1: \theta \neq 0.5$. Two-sided test.
\item<2->[] \color{red} Test-statistic: $X$ is the number of heads tossed. $x = 1070$.
\item<2->[] \color{red} Under $H_0$, $\mu = n\theta = 2000(0.5) = 1000$ and $\sigma^2 = n\theta(1-\theta) = 500$.
\item<2->[] \color{red} We calculate $p\text{-value} = 2P(X \geq 1070)$ since 1070 is greater than the mean.
\vspace*{-1ex}
\begin{align*}
p\text{-value} = 2P(X \geq 1070) &= 2P\bigg(\frac{X-1000}{\sqrt{500}} \geq \frac{1070-1000}{\sqrt{500}}\bigg)\\
&= 2P(Z \geq 3.13)\\
&= 2(1-0.9991) \quad \text{from the z-chart}\\
&= 0.0018
\end{align*}
\vspace*{-5ex}
\item<2->[] \color{red} $p$-value $< 0.05$, so we reject $H_0$.
\end{itemize}
\end{frame}

\begin{frame}{Example}
\begin{itemize}
\setlength{\itemsep}{10pt}
\item I want to test the hypothesis a coin is unbiased. I observed 1,070 heads out of 2,000 tosses. Let $\theta$ be the probability of tossing a head. How would the approach be different if we were looking at the \textit{proportion}?
\item<2->[Ans:] \color{red} $H_0: \theta = 0.5$ vs. $H_1: \theta \neq 0.5$.
\item<2->[] \color{red} Test-statistic: $P$ is the proportion of heads tossed. $p = 0.535$.
\item<2->[] \color{red} Under $H_0$, $\mu = \theta = 0.5$ and $\sigma^2 = \frac{\theta(1-\theta)}{n} = \frac{1}{8000}$.
\item<2->[] \color{red} Do the same as before...
\end{itemize}
\end{frame}



\end{document}


